% !TEX TS-program = xelatex
% !TEX encoding = UTF-8 Unicode
% !Mode:: "TeX:UTF-8"

\documentclass{resume}
\usepackage{zh_CN-Adobefonts_external} % Simplified Chinese Support using external fonts (./fonts/zh_CN-Adobe/)
%\usepackage{zh_CN-Adobefonts_internal} % Simplified Chinese Support using system fonts
\usepackage{linespacing_fix} % disable extra space before next section
\usepackage{cite}

\begin{document}
\pagenumbering{gobble} % suppress displaying page number

\name{马晓龙}

\basicInfo{
  \email{mxl.m@qq.com} \textperiodcentered\ 
  \phone{(+86) 15600519843} \textperiodcentered\ 
%  \github[DragonFive]{https://github.com/DragonFive}}
  \linkedin[xiaolong-ma]{https://www.linkedin.com/in/xiaolong-ma-06501bb6}}
 
\section{\faGraduationCap\ 教育背景}
\datedsubsection{\textbf{中科院计算所}, 北京}{2015 -- 至今}
\textit{在读硕士研究生}\ 计算机视觉/深度学习, 预计 2018 年 6 月毕业
\datedsubsection{\textbf{西安电子科技大学}, 西安, 陕西}{2011 -- 2015}
\textit{学士}\ 软件工程 (全院排名5\%)

\section{\faUsers\ 实习经历}
\datedsubsection{\textbf{商汤科技} 北京}{2017年04月 -- 至今}
\role{见习算法研究员}{部门: 算法平台部}
基于多机多卡的深度学习框架sensenet开发,在虚拟化集群上的docker镜像开发和DL模型训练工作
\begin{itemize}
  \item 基于caffe开发用于高性能GPU集群的多机多卡优化的深度学习框架sensenet;
  \item 开发sensenet的docker镜像,提供GPU支持和rdma网络支持
  \item 部署kubenetes虚拟化集群,编写api提供一键式多机多卡训练网络模型,并进行集群资源监控
  \item 工具:caffe,mpi,cuda,memcached,kubenetes,docker,slurm,heapster等
\end{itemize}

\datedsubsection{\textbf{理光中国}北京}{2016年03月 -- 2016年09月}
\role{计算机视觉实习生}{部门:视频算法研究部}
街景视频浓缩系统开发
\begin{itemize}
  \item 使用VIBE算法在视频中背景建模,使用faster rcnn进行目标的检测
  \item 搭建视频浓缩系统,实现24倍视频浓缩,目标结构化存储检索,设计自适应目标叠加算法
  \item 视频浓缩以缓存和多线程方式对回贴算法提速900倍
\end{itemize}

\section{\faCogs\ 项目经历}
\datedsubsection{\textbf{物端智能视频监控课题}}{2016年10月 -- 至今}
\role{项目策划与算法研究}{毕设课题}
%\begin{onehalfspacing}
搭建一套视频监控平台,实现运动目标检测与跟踪、深度学习模型压缩优化、视频结构化存储与检索
\begin{itemize}
  \item 设计组合特征与BOW模型实现目标结构化存储与检索
  \item 独立研究TLD/MDNet/GoTurn等跟踪算法,改进MDNet网络结构,拟发表论文
  \item 用squeezenet/mobilenet模型和deep compression算法在arm平台上对跟踪算法压缩优化
\end{itemize}
%\end{onehalfspacing}


\datedsubsection{\textbf{海上舰船检测与匹配}}{2014年11月 -- 2016年06月}
\role{算法设计与程序开发}{重大专项}
图片中舰船的检测以及匹配,视频中运动船只的检测统计
\begin{itemize}
  \item sift特征进行海上舰船图像匹配,制作船只图像数据集
  \item 高斯混合模型GMM和腐蚀膨胀方法检测视频运动船只,监控船只流量
  \item 使用adaboost与harr特征,svm算法hog特征,DPM算法以及faster rcnn进行船只检测
\end{itemize}


% Reference Test
%\datedsubsection{\textbf{Paper Title\cite{zaharia2012resilient}}}{May. 2015}
%An xxx optimized for xxx\cite{verma2015large}
%\begin{itemize}
%  \item main contribution
%\end{itemize}


\section{\faInfo\ 专业技能}
% increase linespacing [parsep=0.5ex]
\begin{itemize}[parsep=0.5ex]
  \item github: https://github.com/DragonFive 技术博客: https://dragonfive.github.io/
  \item 熟悉C++/Python等语言,opencv/caffe/cuda等库,k8s/docker等技术和mpi/slurm等集群工具
  \item 熟悉集群架构、DL框架并行、CNN网络模型、模型压缩优化、目标检测跟踪算法、图像处理
  \item 实现ML算法:https://github.com/DragonFive/python\_cv\_AI\_ML 
  \item 学习cs231N,Deep learning(Yoshua Bengio),Neural Network and Deep Learning(Michael Nielsen)
\end{itemize}

%% Reference
%\newpage
%\bibliographystyle{IEEETran}
%\bibliography{mycite}
\end{document}
